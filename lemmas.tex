\documentclass[11pt]{article}

\usepackage[english]{babel}
\usepackage{amsmath}
\usepackage{graphicx}
\usepackage{amsmath,proof,amsthm,amssymb,color}

\newcommand{\nat}{\mathbb{N}}
\newcommand{\reals}{\mathbb{R}}
\newcommand{\boxinate}[1]{\framebox[1.1\width]{#1}}
\newcommand{\blackslash}{\text{\textbackslash}}
\newcommand{\tand}{~\land~}
\newcommand{\tor}{~\lor~}
\newcommand{\st}{~.~} % lol it's a face
\newcommand{\diam}{\mathtt{diam}}
\newcommand{\prop}{\item \textbf{prop}: }
\newcommand{\defn}{\item \textbf{def}: }
\newcommand{\cor}{\item \textbf{cor}: }
\newcommand{\mathboxinate}[1]{\text{\boxinate{$#1$}}}

\begin{document}

\begin{center}
  {\LARGE 21-355 Lemmas} \\
    \today
\end{center}

\section{Initial Theorems/Definitions}
These are things we should be fairly comfortable with.
\begin{itemize}
\item Defn: $X$ a metric space. If $\forall E \subseteq X, E \neq \emptyset$
  and $E$ bounded from above, then $\sup E \in X$
\item $\forall\;r_1,r_2.\; \exists\;q.\; r_1 < q < r_2$
\item Cauchy Schwartz: $|\langle x, y \rangle| \leq ||x|| ||y||$
\item Union of countable sets is countable. Infinite subset of countable set is
  countable.
\item Limit point: $p$ is a limit point if $\forall\;r > 0.~\exists~q \neq p, q
  \in N_r(p) \land q \in E$
\item Closed: E is closed if $\forall~p$ a limit point of E, $p \in E$
\item Interior Point: $p$ is an interior point in E if $\exists~r > 0.~ N_r(p)
  \subseteq E$
\item Open: $E$ is open if every point is an interior point
\item Sequences must be countable.
\item $S$ has the least upper bound property if $\forall\;E \neq \emptyset, E
  \subseteq S$, E is bound above and $\sup~E \in S$
\item Every neighbourhood is open
\item $\{ G_n \}$ open $\implies \bigcup_n G_n$ open
\item $\{ G_n \}$ closed $\implies \bigcap_n G_n$ closed
\item $\{ G_n \}_{n=1}^{N}$ open $\implies \bigcap_{n = 1}^N G_n$ open
\item $\{ G_n \}_{n=1}^{N}$ closed $\implies \bigcap_{n = 1}^N G_n$ closed
\end{itemize}

\section{Sequences} % Brian

\begin{itemize}
\item $\{p_i\} $ \boxinate{converges} to $p$ $ \iff \exists p \in X . \forall \epsilon
  > 0 . \exists n_0 \in \nat . \forall n \geq n_0 . d(p_n, p) < \epsilon$.

\item $\{p_i\}$ is \boxinate{bounded} $ \iff \exists p \in X . \exists m > 0
  . \forall n \in \nat . d(p_n, p) < m$

\item Some propositions about convergence.
  \begin{itemize}
  \item $p_n \rightarrow p \in X \iff \forall r > 0 . |N_r(p) \blackslash
    \{p_n\}|$ is infinite.


  \item $\{p_i\}$ converges $\implies \{p_i\}$ is bounded.

  \item $E \subset X \tand p$ is a limit point of $E \implies \exists \{p_i\}
    \subset E . p_n \rightarrow p$.

  \end{itemize}

\item If $\{n_i\}$ is a sequence in $\nat$ such that $\forall i \in \nat . n_i
  < n_{i+1}$, then $\{p_{n_i}\}$ is called a \boxinate{subsequence} of
  $\{p_i\}$.

\item If $\{p_{n_i}\}$ converges, its limit is called a
  \boxinate{subsequential limit}.

\item $p_i \rightarrow p \iff \forall \{p_{n_i}\} . p_{n_i} \rightarrow p$.

\item (Bolzano Weirstrauss) $X$ is compact and $\{p_i\} \subseteq X \implies \exists \{p_{n_i}\}
  \subseteq X . \exists p \in X . \{p_{n_i}\} \rightarrow p$.

\item $\{p_i\} \subseteq X$ is a \boxinate{Cauchy sequence} $\iff \forall
  \epsilon > 0 . \exists N_\epsilon . \forall m, n > N_\epsilon . d(p_m, p_n) <
  \epsilon$.

\item $E \subseteq X$, the \boxinate{diameter} of $E$ is
  $$\diam(E) = \sup_{p, q \in E} d(p, q)$$

\item $\diam(\overline{E}) = \diam(E)$
\item $K_n \supseteq K_{n+1} . K_n \neq \emptyset . K_n$ is compact,
  $\lim_{n\rightarrow \infty} \diam(K_n) = 0$, the
  $$\bigcap_{n=1}^\infty K_n$$ contains exactly 1 point.

\item Every convergent sequence is Cauchy.

\item Every Cauchy sequence in a compact metric space $X$ converges to some
  point $p \in X$.

\item Every Cauchy sequence in $\reals^n$ converges.

\item A metric space in which every Cauchy sequence converges is
  \boxinate{complete}.

\item Compact $\implies$ complete.

\item If $\{a_i\}$ is monotonic, it's bounded $\iff$ it converges.

\item Let $\{a_i\} \subseteq \reals^n$ Let
  $$E = \{x \in \reals \cup \{+\infty, -\infty\} : \lim_{k\rightarrow \infty}
  a_{n_k} = x\}$$

  Define

  $$\sup E = \overline{\lim_{n\rightarrow\infty}} a_n = \limsup_{n\rightarrow
    \infty} a_n$$

  $$\inf E = \underline{\lim_{n\rightarrow\infty}} a_n = \liminf{n\rightarrow
    \infty} a_n$$

\item $\limsup a_n \in E$.

\item $x > \limsup a_n \implies \exists n \in \nat . \forall i \geq n . a_i <
  x$.

\item $p \in \reals^+ \tand a \in \reals \implies$
  $$\lim_{n \rightarrow \infty} \frac{n^\alpha}{(1+p)^n}$$

\item $\{a_n\} \subseteq [0, \infty)$.
  $$\sum_{n \in \nat} a_n\text{ converges }\iff \{\sum_{i=1}^n a_i\} \text{ is
    bounded}$$

\item $a_1 \geq a_2 \geq a_3 \geq \ldots$ and $0 \leq a_i$.
  $$\sum_{k \in \nat} a_k \text{ converges } \iff \sum_{k \in \nat} 2^k a_{2^k}
  \text{ converges}$$

\end{itemize}

%%%%%%%%%%%%%%%%%%%%%%%% end of Brian

\section{Compact Metric Spaces} % Rokhini
\begin{itemize}
\item Defn: $\{G_{\alpha}\}$ all open subsets of $X \land E \subseteq \bigcup
  G_{\alpha} \land E \subseteq X \Rightarrow \{G_{\alpha}\}$ is an
  \boxinate{open cover} of $E$

\item Defn: $\forall\;\{G_{\alpha}\}$ such that $E \subseteq \bigcup G_{\alpha}
  ,\;(\exists\;n.\; E \subseteq \bigcup_{\alpha = 1}^n G_{\alpha} \Rightarrow
  E$ is compact)
\item Prop: Closed subsets of compact sets are compact
\item Prop: $E \subseteq X$ where $X$ compact and $|E| = \infty$. Then $E$
  contains a limit point (Heine-Borel)
\item Defn: $A, B \subseteq X$ are separated if $A \cap \overline{B} = B \cap
  \overline{A} = \emptyset$. (Connected = Not separated)
\item $E \subseteq \mathbb{R} \iff ~\forall~x,y \in E$, if $z \in (x, y)$ then
  $z \in E$
\item Prop: $\{I_k\}$ a sequence such that $I_{k+1} \subseteq I_k \forall\;k
  \Rightarrow \bigcap I_k \neq \emptyset$
\item Defn: An $n$ cell is an n-dimensional hypercube defined on $\mathbb{R}^n$
\item Prop: Every ball in $\mathbb{R}^n$ contains an $n$ cell and every $n$
  cell contains a ball
\item Prop: $n$ cells are compact (Proved by contradiction using nested
  quadrants as intervals)
\item Prop: $\{K_{\alpha}\}$ compact and non-empty $\land\; \bigcap_{\alpha =
  1}^n K_{\alpha} \neq \emptyset \Rightarrow \bigcap_{\alpha}^{\infty}
  K_{\alpha} \neq \emptyset$
\item Corollary: $\forall\;j, K_j$ compact and non-empty $\land \;K_{j+1}
  \subseteq K_j \Rightarrow \bigcap_{j=1}^{\infty} K_j \neq
  \emptyset$. Eg. Cantor set
\item Defn: Let $E \subseteq Y \subseteq X$. $E$ is relatively open wrt $Y$ if
  $\forall\;x \in E.\;\exists\;r > 0.\;\{y \in Y: d(x, y) < r\} \subseteq E$
\end{itemize}

\section{Continuity}

\begin{itemize}

\item Let X,Y,Z be metric spaces and $\exists$  X

\item F is said to be $\boxinate{continuous}$ at p $\exists$ if $\forall
  \epsilon >$0, such that if $d_{x}$(x,p) $< \delta$ and x $\in$ X, , then
  $d_y$(F(x),F(p)) $< \epsilon$


\item If F: E $\rightarrow$  Y and G: F(E) $\rightarrow$ Z and h(x) = g(f(x))
  If F,G is continuous at at a point P, then h is cont. at p



\item Prop: F : X $\rightarrow$ Y is cont. $\iff$ $F^{-1}$(V) open for V open


\item Corollary: F is cont. $\iff$ $F^{-1}$(E) is closed if E is closed

\item Let F,G be complex cont. functions in X. Then f+g, fg, $\frac{f}{g}$(g
  $\neq$ 0) are cont. functions.


\item Prop: Let Y be MS, X CMS, if F: X $\rightarrow$ Y is cont., F(X) is
  compact


\item Corollary: X CMS at F: X $\rightarrow \mathbb{R}^k$, then F(x) is closed
  and bounded.

\item Suppose F is a cont. real function on a CMS X and

  $$M = \sup_{p \in X}f(p)$$

  $$m = \inf_{p \in X}f(p)$$

\item Prop: If F: X (CMS) $\rightarrow$ Y (CMS) is a cont. bijection, then
  $F^{-1}$ is cont.

\item Prop: Let F: X (CMS) $\rightarrow$ Y (MS) be a cont. function. Then F is
  uniformly continuous.

\item Prop: Let F: X (CMS) $\rightarrow$ Y be continuous, and E $\subseteq$ X
  be connected. Then F(E) is connected.

\item Corollary: Let F: [a,b) $\rightarrow \mathbb{R}$ be a cont. function. If
  F(a) $<$ F(b) then $\forall$ y s.t. F(a) $<$ y $<$ F(b), $\exists$ c $\in$
  (a,b) such that y = F(c) [IMVT]

\item Prop: Let f be monotonically increasing on (a,b) and x $\in$ (a,b) Then:

  $$ \sup_{a < t < x} f(t) = f(x-) \leq f(x) \leq f(x+) = \inf_{x < t < b}
  f(t) $$

\item Corollary: Monotonic functions have at most a countable set of jump
  discontinuities [Monotonic functions only have jump disconts.?]

\end{itemize}

\section{Derivative}

\begin{itemize}
  % notes from 2/12/14 begin ------>
  \defn $(c, +\infty)$ is a \boxinate{neighborhood of $+\infty$}.

  \defn $(-\infty, c)$ is a \boxinate{neighborhood of $-\infty$}.

  \defn $f : E \subseteq \reals \rightarrow \reals$. \\
    $t, x \in [-\infty, \infty]$. \\
    We say \boxinate{$f(t) \rightarrow a$ as $t \rightarrow x$} $\iff$
    $$\forall \text{ neighborhood } U \text{ of a} \st \exists \text{
      neighborhood } V \text{ of } x \st V \cap E \neq \emptyset \tand (\forall
    t \in V \cap E \st f(t) \in U)$$

    For example, $f(t) = t^2$ approaches $x = +\infty$. If we have a
    neighborhood $U = (c, +\infty)$, then $V = (|c|^{frac 1 2}, +\infty)$ will
    satisfy that for $t \in V \cap E$, $f(t) \in U$.

  \defn $f' : (a, b) \rightarrow \reals$ is the \boxinate{derivative} of $f
  \iff$

  $$f'(x) = \lim_{t \rightarrow x} \frac{f(t) - f(x)}{t-x}$$

  If the derivative exists at $x \in (a, b)$, then we say that $f$ is
  \boxinate{differentiable} at $x$.

  \defn $f : X \rightarrow \reals \st f$ has a \boxinate{local maxima} a $p \in
  X \iff$

  $$\exists \delta > 0 \st \forall q \in X \tand d(p, q) < \delta \st f(q)
  \leq f(p)$$

  \prop If $f$ has a local maxima at $p \in X$ and $f'(x)$ exists, then $f'(x)
  = 0$.

  \prop $f$ and $g$ are differentiable on $(a, b) \implies$

  $$\exists x \in (a, b) \st (f(b) - f(a))g'(x) = (g(b)-g(a))f'(x)$$

  \begin{itemize}
    \cor $f$ is differentiable on $(a, b) \implies$

    $$\exists x \in (a, b) \st f(b) - f(a) = f'(x)(b - a)$$

    Think intermediate value theorem. There exists a point at which the slope
    equals the average slope.
  \end{itemize}
  % notes from 2/12/14 end <------

\end{itemize}
\section{Integration}
  % notes from 2/24/14 start ------>
\begin{itemize}
  \prop $f : [a,b ] \rightarrow \reals^n$ is continuous, and differentiable on
  $(a, b)$ implies

  $$\exists x \in (a, b) \st |f(b) - f(a)| \leq |f'(x)| (b-a)$$

  \defn A \boxinate{partition} $P$ of $[a, b]$ is finite set $\{x_0, \ldots
  x_n\}$ such that $a = x_0 < x_1 < x_2 < \ldots < x_n = b$.

  \defn
  \begin{align*}
    \mathboxinate{M_i} &= \sup_{x_{i-1} \leq x \leq x_i} f(x) \\
    \mathboxinate{m_i} &= \inf_{x_{i-1} \leq x \leq x_i} f(x) \\
    \mathboxinate{U(P, f)} &= \sum_{i = 1}^n M_i \cdot \Delta x_i \\
    \mathboxinate{L(P, f)} &= \sum_{i = 1}^n m_i \cdot \Delta x_i
  \end{align*}

  \defn

  \boxinate{upper Riemann integral} is

  $$\overline{\int_a^b} f(x) dx = \inf \{U(P,f) : P \text{ is a partition of }
  [a,b]\}$$

  \boxinate{lower Riemann integral} is

  $$\underline{\int_a^b} f(x) dx = \sup \{L(P,f) : P \text{ is a partition of }
  [a,b]\}$$

  If the upper Riemann integral = lower Riemann integrl, then we say $f$ is
  Reimann integrable.

  \defn

  For $\alpha : [a, b] \rightarrow \reals$ is monotonically increasing,

  $$\mathboxinate{\Delta\alpha_i} = \alpha(x_i) - \alpha(x_{i-1})$$

  Let

  \begin{align*}
    U(P, f, \alpha) &= \sum_{i=1}^n M_i \cdot \Delta_i \\
    L(P, f, \alpha) &= \sum_{i=1}^n m_i \cdot \Delta_i
  \end{align*}

  \defn

  \boxinate{upper Riemann-Stieltgies integral} is

  $$\overline{\int_a^b} f(x) dx = \inf_P U(P, f, \alpha)$$

  \boxinate{lower Riemann-Stieltgies integral} is

  $$\underline{\int_a^b} f(x) dx = \sup_P L(P, f, \alpha)$$

  If they are equal, we write $\mathboxinate{f \in \Re(\alpha)}$.
  % notes from 2/24/14 end <------
  
  


\end{itemize}

%Anik's notes on Frenet 



\section{Frenet}

\begin{itemize}

\item

\underline{Defitinion:} A curve $\alpha$: [a,b] $\rightarrow \mathbb{R}^n$ is said to be $regular$ if $\alpha'(t) \neq 0 \; \forall t \in [a,b]$

\item A reparametrization of a curve $\gamma$ is a curve $\alpha(t) = \gamma(\phi(t))$ where $\phi$ : [c,d] $\rightarrow$ [a,b] is a smooth bijection.

\item \underline{Proposition:} Any smooth, regular curve $\gamma$ can be reparametrized to have constant speed (if this is true, we say that $\gamma$ is parametrized by arc length)

\item \underline{Theorem:} Let $\alpha$ be a closed regular, smooth curve enclosing on area A. Then, $4\pi A \leq L^2$, where L is the length of $\alpha$.

\item For a fixed curve $\alpha(t)$, the following are the Frenet Formulas: \\ \\

\underline{Unit Tangent Vector:} 

$$ T(t) = \frac{\alpha ^{'}(t)}{||\alpha ^{'}(t)||} $$

\underline{Normal Vector:}

$$ N(t) = \frac{T ^{'}(t)}{||T ^{'}(t)||} $$

\underline{Curvature:}

$$ K(t) = \frac{||T ^{'}(t)||}{||\alpha ^{'}(t)||} $$

\underline{Binormal Vector:}

$$ B(t) = T(t) \; \text{X} \; N(t) $$

\underline{Torsion :}

$$ \tau(t) = N^{'}(t)*B(t) = -N(t)*B^{'}(t) $$


\item A space curve is a line $\iff$ its curvature = 0

\item A space curve is planar $\iff$ its torsion = 0.

Morover, the only planar curves with non-zero constant curvature are portions of circles




  % notes from 2/26/14 start ------>

  \defn $P^*$ is a \boxinate{refinement} of $P \iff P^* \supset P$.

  % some fairly obvious lemmas follow about refinements

  \prop

  $$f \in \Re(\alpha) on [a,b] \iff \forall \epsilon > 0 \st \exists P \st
  |U(P, f, \alpha) - L(P, f, \alpha)| < \epsilon$$

  \prop continuous $\implies$ integrable on any $\alpha$.
  % notes from 2/26/14 end <------

  % notes from 2/28/14 start ------>
  \prop $f$ is monotonically increasing $\implies \forall \alpha$ continuous
  $\st f \in \Re(\alpha)$.

  \prop \\
  $f$ is bounded on $[a, b]$, and \\
  $f$ has finitely many points of discontinuity, and \\
  $\alpha$ is continuous at the discontinuities of $f$, \\
  $\implies f \in \Re(\alpha)$.

  \prop \\
  $f \in \Re(\alpha)$ on $[a, b]$, and \\
  $m \leq f(x) \leq M$, and \\
  $\phi$ is continuous on $[m, M]$, \\
  $\implies \phi \circ f \in \Re(\alpha)$

  % notes from 2/28/14 end <------
\end{itemize}
\end{document}
